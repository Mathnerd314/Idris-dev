\section{Discussion}

\label{sect:discussion}

%Discuss performance (anecdotally, how big is the library and how long to check,
%and relative to previous version). 

\subsection{Related Work}

Dependently typed programming languages have becoming more prominent in recent
years as tools for verifying software correctness, and several experimental
languages are being developed, in particular Agda \cite{norell2007thesis}, Epigram
\cite{McBride2004a,Levitation2010} and Trellys \cite{Kimmell2012}.

An earlier implementation of \Idris{} was built on the \Ivor{} proof engine 
\cite{Brady2006b}. This implementation differed in one important way --- unlike
the present implementation, there was limited separation between the type theory
and the high level language. The type theory itself supported implicit syntax and
unification, with high level constructs such as the \texttt{with} rule implemented
directly. Two important disadvantages were found with this approach, however: 
firstly, the type checker is much more complicated when combined with
unification, making it harder to maintain; secondly, adding new high level features
requires the type checker to support those features directly. In contrast, elaboration
by tactics gives a clean separation between the low level and high level languages.

The Agda implementation is based on a type theory with
implicit syntax and pattern matching --- Norell gives an algorithm for type checking
a dependently typed language with pattern matching and metavariables 
\cite{norell2007thesis}. Unlike the present \Idris{} implementation, metavariables
and implicit arguments are part of the type theory. This has the advantage that
implicit arguments can be used more freely, at the expense of complicating the type
system.

Epigram \cite{McBride2004a} and Oleg \cite{McBride1999} 
have provided much of the inspiration for the \Idris{} elaborator. Indeed,
the hole and guess bindings of \TTdev{} are taken directly from Oleg.
\Epigram{} does not implement pattern matching directly, but rather translates
pattern matching into elimination rules \cite{McBride2002}. This has the
advantage that
elimination rules provide termination and coverage proofs \emph{by construction}.
Furthermore they simplify implementation of the evaluator and provide easy
optimisation opportunities \cite{Brady2003}. However, it requires the
implementation of extra machinery for constructor manipulation
\cite{McBride2006} and so we have avoided it in the present implementation.

%Comparison with GHC's type system and type checker. \cite{Vytiniotis2011}

%[Observation: separate elaboration and type checking, sort of like in GHC which
%type checks the high level language and produces a type correct core language.
%Elaboration is effectively a type checker for the high level language, so we have
%a hope of providing reasonable error messages related to the original code.]

\subsection{Conclusion}

In this paper, I have given an overview of the programming language \Idris{},
and its core type theory \TT{}, giving a detailed algorithm for translating
high level programs into \TT{}.
\TT{} itself is deliberately small and simple, and the design has deliberately
resisted innovation so that we can rely on existing metatheoretic properties
being preserved. The kernel of the \Idris{} implementation consists of a type checker
and evaluator for \TT{} along with a pattern match compiler, which are implemented
in under 1000 lines of Haskell code. It is important that this kernel remains small
--- the correctness of the language implementation relies to a large extent on
the correctness of the underlying type system, and keeping the implementation small
reduces the possibility of errors.

The approach we have taken to implementing the high level language, implementing
a tactic language as an embedded DSL, allows us to
build programs on top of a small and unchanging kernel, rather than extending 
the core language to deal with implicit syntax, unification and type classes.
High level \Idris{} features are implemented by describing the corresponding
sequence of tactics to build an equivalent program in \TT{}, via a development
calculus of incomplete terms, \TTdev{}. A significant advantage we have found with
this approach is that higher level features can easily be implemented in terms
of existing elaborator components. For example, once we have implemented elaboration
for data types and functions, it is easy to add several features:

\begin{itemize}
\item \textbf{Type classes}: A dictionary is merely a record containing the
functions which implement a type class instance. Since we have a tactic based
refinement engine, we can implement type class resolution as a tactic.
\item \textbf{\texttt{where} clauses}: We have access to local variables and
their types, so we can
elaborate \texttt{where} clauses at the point of definition simply by lifting
them to the top level. 
\item \textbf{\texttt{case} expressions}: Similar to \texttt{where} clauses,
these are implemented by lifting the branches out to a top level function.
\end{itemize}

We do not need to make any changes to the core language type system in order to 
implement these high level features. 
Other high level features such as dependent records, tuples and monad comprehensions
can be added equally easily.  Furthermore, 
since we have used a tactic-based EDSL to elaborate \Idris{} to \TT{}, it is
possible to expose the tactic language to the programmer. This opens up the
possibility of implementing domain specific decision procedures, or implementing
user defined tactics in a style similar to Coq's \texttt{Ltac} language \cite{Bertot2004}.

The objective of this implementation of \Idris{} is to provide a platform
for experimenting with realistic, general purpose programming with dependent
types, by implementing a Haskell-like language augmented with \emph{full}
dependent types.  
In this paper, we have seen how such a high level language can be implemented
by building on top of a small, well-understood, easy to reason about type
theory. However, a programming language implementation is not an end in itself. 
Programming languages exist to support research and practice in many different
domains. In future work, therefore, we plan to apply domain specific
language based techniques to realistic problems
in important safety critical domains such as security and network protocol
design and implementation. In order to be successful, we need a language which
is expressive enough to describe protocol specifications at a high level, and robust 
enough to guarantee correct implementation of those protocols. \Idris{}, we believe,
is the right tool for this work.



%\subsection{Further Work}

%[Would the EDSL approach work in other languages? Adding components of DTP
%to imperative languages, say, using \TT{} as a verified core.
%\Idris{} implementation as the beginning of a project to explore practical
%DTP --- systems, protocols, security especially. And just having full
%dependent types for lightweight correctness guarantees.]
